%-------------------------
% Resume in Latex
% Author : Abey George
% Based off of: https://github.com/sb2nov/resume
% License : MIT
%------------------------

\documentclass[letterpaper,11pt]{article}

\usepackage{latexsym}
\usepackage[empty]{fullpage}
\usepackage{titlesec}
\usepackage{marvosym}
\usepackage[usenames,dvipsnames]{color}
\usepackage{verbatim}
\usepackage{enumitem}
\usepackage[hidelinks]{hyperref}
\usepackage[english]{babel}
\usepackage{tabularx}
\usepackage{fontawesome5}
\usepackage{multicol}
\usepackage{graphicx}
\setlength{\multicolsep}{-3.0pt}
\setlength{\columnsep}{-1pt}
\input{glyphtounicode}

\RequirePackage{tikz}
\RequirePackage{xcolor}
\RequirePackage{fontawesome}
\usepackage{tikz}
\usetikzlibrary{svg.path}


\definecolor{cvblue}{HTML}{0E5484}
\definecolor{black}{HTML}{130810}
\definecolor{darkcolor}{HTML}{0F4539}
\definecolor{cvgreen}{HTML}{3BD80D}
\definecolor{taggreen}{HTML}{00E278}
\definecolor{SlateGrey}{HTML}{2E2E2E}
\definecolor{LightGrey}{HTML}{666666}
\colorlet{name}{black}
\colorlet{tagline}{darkcolor}
\colorlet{heading}{darkcolor}
\colorlet{headingrule}{cvblue}
\colorlet{accent}{darkcolor}
\colorlet{emphasis}{SlateGrey}
\colorlet{body}{LightGrey}



%----------FONT OPTIONS----------
% sans-serif
% \usepackage[sfdefault]{FiraSans}
% \usepackage[sfdefault]{roboto}
% \usepackage[sfdefault]{noto-sans}
% \usepackage[default]{sourcesanspro}

% \usepackage{serif}
% \usepackage{CormorantGaramond}
% \usepackage{charter}



% \pagestyle{fancy}
% \fancyhf{}  % clear all header and footer fields
% \fancyfoot{}
% \renewcommand{\headrulewidth}{0pt}
% \renewcommand{\footrulewidth}{0pt}

% Adjust margins
\addtolength{\oddsidemargin}{-0.6in}
\addtolength{\evensidemargin}{-0.5in}
\addtolength{\textwidth}{1.19in}
\addtolength{\topmargin}{-.7in}
\addtolength{\textheight}{1.4in}

\urlstyle{same}

\raggedbottom
\raggedright
\setlength{\tabcolsep}{0in}

% Sections formatting
\titleformat{\section}{
  \vspace{-4pt}\scshape\raggedright\large\bfseries
}{}{0em}{}[\color{LightGrey}\titlerule \vspace{-5pt}]

% Ensure that generate pdf is machine readable/ATS parsable
\pdfgentounicode=1

%-------------------------
% Custom commands

\newcommand{\resumeItem}[1]{
  \item\small{
    {#1 \vspace{-2pt}}
  }
}

\newcommand{\classesList}[4]{
    \item\small{
        {#1 #2 #3 #4 \vspace{-2pt}}
  }
}

\newcommand{\resumeSubheading}[4]{
  \vspace{-1pt}\item
    \begin{tabular*}{0.97\textwidth}{l@{\extracolsep{\fill}}r}
      \textbf{#1} & #2 \\
      \textit{\small#3} & \textit{\small #4} \\
    \end{tabular*}\vspace{-5pt}
}


\newcommand{\resumeSubSubheading}[2]{
    \item
    \begin{tabular*}{0.97\textwidth}{l@{\extracolsep{\fill}}r}
      \textit{\small#1} & \textit{\small #2} \\
    \end{tabular*}\vspace{-7pt}
}


\newcommand{\resumeProjectHeading}[2]{
    \item
    \begin{tabular*}{1\textwidth}{l@{\extracolsep{\fill}}r}
      \small#1 & \textbf{\small #2}\\
    \end{tabular*}\vspace{2pt}
}

\newcommand{\resumeSubItem}[1]{\resumeItem{#1}\vspace{-4pt}}

\renewcommand\labelitemi{$\vcenter{\hbox{\tiny$\bullet$}}$}
\renewcommand\labelitemii{$\vcenter{\hbox{\tiny$\bullet$}}$}
\renewcommand\labelitemiii{$\vcenter{\hbox{\tiny$\bullet$}}$}

\newcommand{\resumeSubHeadingListStart}{\begin{itemize}[leftmargin=0.01in, label={}]}
\newcommand{\resumeSubHeadingListEnd}{\end{itemize}}
\newcommand{\resumeItemListStart}{\begin{itemize}}
\newcommand{\resumeItemListEnd}{\end{itemize}\vspace{-5pt}}


\newcommand\sbullet[1][.5]{\mathbin{\vcenter{\hbox{\scalebox{#1}{$\bullet$}}}}}

%-------------------------------------------
%%%%%%  RESUME STARTS HERE  %%%%%%%%%%%%%%%%%%%%%%%%%%%%


\begin{document}

%----------HEADING----------



\begin{tabular*}{\textwidth}{l@{\extracolsep{\fill}}r}
  \textbf{\Huge Umang Garg} & 
    \href{https://www.linkedin.com/in/umang-garg/}{\raisebox{-0.2\height}\faLinkedinSquare\ \underline{Linkedin}}  ~
    \href{https://github.com/umang-garg21}{\raisebox{-0.2\height}\faGithub\ \underline{Github}} ~\href{https://scholar.google.com/citations?user=uoffd9cAAAAJ&hl=en}{\raisebox{-0.2\height}\faGraduationCap\ \underline{Scholar}}
    \\
  \href{mailto:umang@ucsb.edu}{\raisebox{-0.1\height}\faEnvelope\  umang@ucsb.edu} ~  \small \href{tel:+1 805*7249299}{ \raisebox{-0.05\height}\faPhone\ +1-805-724-9299 ~}  & {\raisebox{-0\height}\faMapMarker\ U.C. Santa Barbara, CA} \\
\end{tabular*}\\

% \vspace{8pt}
% {\color{Gray} \par\noindent\rule{\textwidth}{0.05pt}  }
%---------- SUMMARY-----------
% \vspace{-10pt}

% \raisebox{.2ex}{\color{cvblue}{\normalsize{\textbf{\underline{Summary}}}}}: \normalsize{Summer-Internship $|$ Fall Co-op applicant: Result-driven computer engineer with with numerous software code design experiences. Understand value of patents as crucial assets, holding 3 US patent applications. A team player with good negotiation skills to cater to differing design goals across multiple teams.}


  \vspace{-6pt}
 
 %-----------PROGRAMMING SKILLS-----------
\section{\normalsize{\color{cvblue}{TECHNICAL SKILLS}}}
 \begin{itemize}[leftmargin=0.15in, label={}]
    \small{\item{
     \textbf{\normalsize{Languages:}}{ \normalsize{Python, C/ C++, MATLAB, Verilog/ Systemverilog, Assembly, HTML, RTL }} \\
     
     \vspace{3pt}
     \textbf{\normalsize{Developer Tools/ Frameworks:}}{ \normalsize{Pytorch, IBM CPLEX, Noxim$\mathbf{++}$, Numpy, Pandas, Linux, Vivado Tool Suites, Cadence Virtuoso, DRC, LVS, Simulink, Labview, GIT, Ansys HFSS, Zynq library, FPGAs. }}  \\
    }}
 \end{itemize}
 
\vspace{-12pt}


%-----------PROJECTS-----------
\section{\normalsize{\color{cvblue}{PROJECTS}}}
    \vspace{-5pt}
    \resumeSubHeadingListStart
    
      \resumeProjectHeading
          {\href{https://github.com/umang-garg21/SNN-SDFG-optimal-mapping}{\textbf{\normalsize{Optimal Compiler Design SDNN mapping}} \href{https://github.com/umang-garg21/SNN-SDFG-optimal-mapping}{\raisebox{-0.1\height}\faExternalLink }} $|$ {\underline{Swarm Opt, PyCARL, Noxim++ }}}{Feb `21 - Ongoing}
          \resumeItemListStart

           \vspace{-2pt} \resumeItem{\normalsize{Investigating efficient graph-cut algos (KL) to map SNNs to least-interacting tiles: saving NoC hops, BW.}}
           
           \vspace{2pt}
            \resumeItem{\normalsize{A multi-variate (energy, accuracy, latency, throughput) space search using PSO, Tabu, hill-climb techniques.}}
          \resumeItemListEnd 
          \vspace{-8pt}
    
      \resumeProjectHeading
          {\href{https://github.com/umang-garg21/Pacman_Agent}{\textbf{\normalsize{Rational AI multi-agent design for PACMAN}} \href{https://github.com/umang-garg21/Pacman_Agent}{\raisebox{-0.1\height}\faExternalLink }} $|$ {\underline{Python, DSA, Heuristics }}}{Feb `21 - March `21}
          \resumeItemListStart

           \vspace{-2pt} \resumeItem{\normalsize{ Designed multi-adversary-aware intelligent Pacman AI agent: reflex-based, Minimax, Expectimax models.}}
           
           \vspace{2pt}
            \resumeItem{\normalsize{Goal-tailored heuristic modelling. Implemented greedy search, $\alpha$-$\beta$ pruning, iterative deepening to time-limit.}}
          \resumeItemListEnd 
          \vspace{-5pt}
    
    % \resumeProjectHeading
    %       {\href{https://github.com/UMANG-GARG-UCSB/Naive_Bayes_classifier}{\textbf{\normalsize{Naive Bayes Classifier for Athletic classification}} \href{https://github.com/UMANG-GARG-UCSB/Naive_Bayes_classifier}{\raisebox{-0.1\height}\faExternalLink }} $|$ {\underline{Python, Numpy, Pandas}}}{Jan 2021}
    %       \resumeItemListStart
    %         \resumeItem{\normalsize{Developed classifier to classify different athletes based on multiple features on athletic measurements. }}
            

    %       \resumeItemListEnd 
    %       \vspace{-5pt}
              \vspace{-5pt}
          \resumeProjectHeading
          {\href{https://github.com/umang-garg21/Unsupervised_SNN}{\textbf{\normalsize{Unsupervised SNN Learning for Digit Recognition}} \href{https://github.com/UMANG-GARG-UCSB/Unsupervised_SNN}{\raisebox{-0.1\height}\faExternalLink }} $|$ {\underline{SNNTorch $|$ Verilog}}}{Jan 2021}
          \resumeItemListStart
            
            \vspace{-2pt}
            \resumeItem{\normalsize{Implemented end-to-end neural network in digital RTL-logic with in-situ \textbf{online STDP learning} support.}}
            
            \vspace{2pt}
               \vspace{-2pt}
            \resumeItem{\normalsize{New local unsupervised learning mechanisms Inter-synaptic traces mechanisms tested. Lateral inhibition etc.}}

          \resumeItemListEnd 
          \vspace{-25pt}
    
    \resumeProjectHeading
          {\textbf{\normalsize{Parallel time-domain Compute-in-memory (CIM) Spiking NN}} $|$ {\underline{Pytorch, Virtuoso}}}{Sept - Dec `21}
          \resumeItemListStart
            \resumeItem{\normalsize{Collaborated and developed time-domain parallel spiking paradigm: \textbf{83x EDP improvement} over SOTA.}}
            
            \vspace{2pt}
            
            \resumeItem{\normalsize{Reimagined dataflow for better weight reuse. Tested on Fashion-MNIST, NMNIST, CIFAR-10 datasets.}}

          \resumeItemListEnd 
        
          \vspace{-10pt}
        %   \resumeProjectHeading
        %   {\href{https://github.com/umang-garg21/ECE-Algo-HW-CoDesign-Project}{\textbf{\normalsize{Accelerating Auditory Recognition on edge- FPGA}} \href{{https://github.com/UMANG-GARG-UCSB/ECE-Algo-HW-CoDesign-Project}}{\raisebox{-0.1\height}\faExternalLink }} $|$ {\underline{Verilog, CompArch}}}{Sept - Dec `21}
        %   \resumeItemListStart
        %       \vspace{-2pt}
        %     \resumeItem{\normalsize{Implemented hardware accelerator for audio recognition, achieving \textbf{~10x improvement} over CPU.}}
            
        %     \vspace{2pt}
            
        %     \resumeItem{\normalsize{Employed \textbf{Algo-HW co design} techniques: mixed precision processing, multiplier tree, pipeline, Max pool.}}

        %   \resumeItemListEnd

        %   \vspace{-10pt}
          
      \resumeProjectHeading
          {\href{https://github.com/umang-garg21/TETRIS-on-FPGA}{\textbf{\normalsize{Developing `TETRIS' firmware on Xilinx FPGA}} \href{https://github.com/UMANG-GARG-UCSB/TETRIS-on-FPGA}{\raisebox{-0.1\height}\faExternalLink }} $|$ {\underline{Embedded C, QPNano, Vivado}}}{Sept - Dec `21}
          \resumeItemListStart
               \vspace{-2pt}
            \resumeItem{\normalsize{Used QPNano \textbf{Hierarchical FSM} for designing game states on Xilinx Artix-A7 board on Vivado.}}
            
            \vspace{2pt}
            
            \resumeItem{\normalsize{Interfaced SPI LCDs and push buttons for interactive gameplay; ensured correct \textbf{interrupt handling}.}}

          \resumeItemListEnd 

    \vspace{-10pt}

          \resumeProjectHeading
          {\href{}{\textbf{\normalsize{Hyperspectral Aerial-Vehicle Anomaly detection at Edge }} \href{https://github.com/UMANG-GARG-UCSB/TETRIS-on-FPGA}{\raisebox{-0.1\height}\faExternalLink }} $|$ {\underline{MATLAB, Virtuoso}}}{May '19 - Aug '19}
          \resumeItemListStart
               \vspace{-2pt}
               
       \resumeItem{\normalsize{          Worked on hyperspectral aerial anomaly detection techniques and multi-band flexible-grain filter design.}}
  
        \vspace{2pt}
            
        \resumeItem{\normalsize{ Co-integrated anomaly detection unit with hypersectral imager for system-constrained perception. }}

          \resumeItemListEnd 

          
          
    \resumeSubHeadingListEnd
\vspace{-8pt}

%
%-----------EXPERIENCE-----------
\section{\normalsize{\color{cvblue}{INDUSTRY EXPERIENCE}}}
  \resumeSubHeadingListStart
      
          \resumeSubheading
      {QpiAI Technologies \href{https://qpiai.tech/}{\raisebox{-0.1\height}\faExternalLink } $|$ \underline{\normalfont{\textit{Design Engineer}}}}{Dec `20 -- Jul `21 $|$ \textit{\small{Bangalore, India}}} 
      {}{}
      \vspace{-13pt}
      
      \resumeItemListStart
       \resumeItem{\normalsize{ Developed ``Auxiliary Pulse Cancellation” code, boosting qubit fidelity times by $10$x. \textbf{2 US patents.} }}
      \resumeItem{\normalsize{Filed \textbf{US patent} proposing an extensively scalable in-silco solution for magnetic field control for qubits. }}
        \resumeItem{\normalsize{Designed a rail-to-rail cryogenic Variable gain amplifier: deployed as \textbf{standalone IP} for qubit control.}}
 
   
        
       
  \resumeSubHeadingListEnd
\vspace{-12pt}


%-----------EDUCATION-----------
\section{\normalsize{\color{cvblue}{EDUCATION}}}
  \resumeSubHeadingListStart
    \resumeSubheading
      {University of California, Santa Barbara (UCSB)}{Sept 2021 -- Present}
      { M.S. in Computer Engineering, Dept. of ECE; GPA: 3.90}{Santa Barbara, CA}
  \resumeSubHeadingListEnd
  \vspace{-8pt}
  \resumeSubHeadingListStart
     \resumeSubheading
      {Birla Institute of Science and Technology, Pilani}{Aug 2016 --  May 2020}
      {B.E. in Electronics and Instrumentation; GPA - 8.4}{Pilani, India}
  \resumeSubHeadingListEnd

% - Literature study on aerial anomaly detection and image recognition techniques
% - Used MATLAB for filter size design for a better target detection
% - Worked to deploy and merge detection at the image-sensor edge with fine-grain adjustments accounting for system constraints
 \vspace{-10pt}


%------RELEVANT COURSEWORK-------
\section{\normalsize{\color{cvblue}{RELEVANT COURSEWORK}}}
    % %\resumeSubHeadingListStart
    %     \begin{multicols}{4}
    %         \begin{itemize}[itemsep=-2pt, parsep=5pt]
    %             \item Artificial Intelligence
    %             \item HW-Algorithm Co-Design for AI Tasks
    %             \item Spiking Neural Networks
    %             \item Embedded Systems Development
    %             \item Probabilistic Computing
    %             \item Deep Neural Networks
    %             \item StatMech. to Quantum Computing 
    %             \item Analog and Digital VLSI Design
    %             \item Computer Architecture
    %             \item RF Microelectronics
    %             \item Antenna Theory and Design

    %         \end{itemize}
    %     \end{multicols}
    %     \vspace*{2.0\multicolsep}
    % %\resumeSubHeadingListEnd

\begin{itemize}[leftmargin=0.15in, label={}]
\vspace{2pt}
    \small{\item{
     \textbf{\normalsize{Graduate:}}{ \normalsize{Data Structures and Algorithms, Artificial Intelligence, Deep Neural Networks, NP-hard Optimizations, Software-HW Co-Design,  Embedded Systems, Neuromorphic Computing, Probabilistic Computing}} \\
    }}
    \vspace{1pt}
    \small{\item{
     \textbf{\normalsize{UnderGraduate:}}{ \normalsize{Analog and Digital VLSI Design, Computer Architecture, RF Microelectronics}} \\
    }}
    
 \end{itemize}
 \vspace{-15pt}


%------Patents and Publications---------

\section{\normalsize{\color{cvblue}{\href{https://scholar.google.com/citations?user=uoffd9cAAAAJ&hl=en}{PATENTS and PUBLICATIONS}}}}

% I filed \textbf{3 US patents} during my short 6-month stint at QpiAI. I have published in 3 top-tier conferences and 1 high-impact journal related to emerging unsupervised ML learning algorithms (contrasitive STDP).
Publication [1]. {Time-domain Parallel Compute-in-memory Spiking Neural Network Architecture and acceleration }

% \href{https://scholar.google.com/citations?user=uoffd9cAAAAJ&hl=en}{Patent [1].} Method and System for generating and regulating local magnetic field variations for spin-qubit manipulation using microstructures in Integrated circuits - \href{https://scholar.google.com/citations?user=uoffd9cAAAAJ&hl=en}{US Appl. Number - 202141007818}.\\
% \vspace{5pt}

\href{https://scholar.google.com/citations?user=uoffd9cAAAAJ&hl=en}{Patent [2].} Method and System for designing hybrid quantum-classical architecture (Q-arc) in quantum computers for individual qubit control in distributed fashion. \textit{(\underline{\href{https://scholar.google.com/citations?user=uoffd9cAAAAJ&hl=en}{3 additional major-Conference publications and 2 more patents}})} 

%-----------INVOLVEMENT---------------
% \section{\normalsize{EXTRACURRICULAR}}
%     \resumeSubHeadingListStart
%         \resumeSubheading{Organization Name \href{Certificate Proof link}{\raisebox{-0.1\height}\faExternalLink } }{MM YYYY -- MM YYYY}{\underline{Role}}{Location}
%             \resumeItemListStart
%                 \resumeItem{\normalsize{About the role \textbf{and responsibilities carried out.}}}
%                 \resumeItem{\normalsize{Participation Certificate. \href{ParticipationCertificateLink.com}{\raisebox{-0.1\height}\faExternalLink }}}
%             \resumeItemListEnd
%     \resumeSubHeadingListEnd
%  \vspace{-11pt}
 
 %-----------CERTIFICATIONS---------------
% \section{\normalsize{CERTIFICATIONS}}

% $\sbullet[.75] \hspace{0.1cm}$ {\href{certificateLink.com}{ReactJS \& Redux - Udemy}} \hspace{1.6cm}
% $\sbullet[.75] \hspace{0.1cm}$ {\href{certificateLink.com}{Java}} \hspace{2.59cm}
% $\sbullet[.75] \hspace{0.2cm}${\href{certificateLink.com} {Command Line in Linux - Coursera}}\\

% $\sbullet[.75] \hspace{0.2cm}${\href{certificateLink.com}{Python for Data Science - XIE}} \hspace{1cm}
% $\sbullet[.75] \hspace{0.1cm}$ {\href{certificateLink.com}{SQL}} \hspace{2.6cm}
% $\sbullet[.75] \hspace{0.2cm}${\href{certificateLink.com}{Microsoft AI Classroom - Microsoft}} \\

% $\sbullet[.75] \hspace{0.2cm}${\href{certificateLink.com}{\textbf{5 Stars} in \textbf{C++} \& \textbf{SQL} \href{certificateLink.com}{\raisebox{-0.1\height}\faExternalLink }}}\hspace{1.45cm}
% $\sbullet[.75] \hspace{0.2cm}${\href{certificateLink.com}{MongoDB Basics}} \hspace{0.5cm}
% $\sbullet[.75] \hspace{0.2cm}${\href{certificateLink.com}{NodeJS with Express \& MongoDB - Udemy}} \\


\end{document}
